\documentclass[conference]{IEEEtran}
\IEEEoverridecommandlockouts
\usepackage{algorithmic}
\usepackage[backend=bibtex,style=ieee,natbib=true]{biblatex}
\usepackage{graphicx}
\usepackage{textcomp}
\usepackage{xcolor}

\addbibresource{main.bib}

\begin{document}

\title{
    COMPSCI 402 --- Artificial Intelligence\\
    Final Report\\
    The Development and Outlook of AI Techniques in The Field of
    Practical Federated Learning on Mobile Devices
}

\author{
    \IEEEauthorblockN{Sichang He}\\
    sichang.he@dukekunshan.edu.cn
}

\maketitle

\begin{abstract}
% TODO: Insert a very brief paragraph to summarize your essay
\end{abstract}

\section{Introduction}

% TODO: Briefly introduce your understanding of AI,

% provide an overview of the application of the AI in your research area.
% For example, you can discuss the current development trend and
% provide a roadmap of the development of AI techniques in your research area.

Federated learning (FL) is a machine learning (ML) technique that
allows edge devices to collaborately train a shared ML model using
local data~\cite{mcmahan2017communication}.
These private data are kept local,
therefore FL is suitable for scenarios where
the training data cannot be shared to a central server due to privacy concerns.

The increasing need to train ML models for business,
the rising awareness of data privacy among individuals,
and the improving laws of governments contribute to
the growing applications of FL.
FL is a compelling option for companies to train ML models using users' data
without breaking some of the privacy laws.
Applications include but are not limited to item ranking, content suggestions,
and next-word prediction~\cite{bonawitz2019towards}.

Unlike model personalization,
where local models are adjusted using each user's data to
adapt to the each specific user,
FL aims to train a global model.
This means that, for example,
while it is common to only update the last layer of neural networks in
model personalization,
we usually want to update most or all layers in FL.
This causes extra restrictions and difficulties.
For example, by only updating the last layer,
model personalization can remain efficient even when
the neural network used is deep;
in contrast, the computational complexity in back propagation for
deep neural networks makes it unsuitable for FL on mobile devices.
Therefore, FL either requires smaller models to be used,
or partial updates,
or techniques to be applied so it does not train the whole model.

\section{TODO: Main Body}

TensorFlow Federated~\cite{tensorflow2015-whitepaper} is oriented to
provide infrastructure for production-level FL on smartphones.
It is widely used for simulation for its rich feature including
decentralized simulation using gRPC,
but it does not have on-device training support~\cite{kholod2020open}.

FedML~\cite{he2020fedml} is a FL as a Service (FLaaS).
It uses MNN for its training on Android,
and it has not provided an iOS SDK.

Flower~\cite{beutel2020flower,mathur2021ondevice}
only provides a communication layer and
leaves the training implementation to the users.
For the training implementation,
its Android example depends on TensorFlow Lite and
its iOS example depends on Core ML.

Syft~\cite{ryffel2018generic,Ziller2021,hall2021syft}
proposes a custom training process and
offers implementations for both Android and iOS,
but it is low-performance and only uses CPU.

PaddlePaddle~\cite{ma2019paddlepaddle} by Baidu is
also mainly for simulations oriented to production.
It supports inference on Android.

FATE~\cite{liu2021fate} is mainly for production-level simulations.
TODO: read again.

OpenFL~\cite{patrick2022openfl} is also mainly for simulations.
TODO: read again.

FedScale~\cite{lai2022fedscale} leans towards benchmarking.
It supports Android via a UNIX environment.

Project Florida~\cite{madrigal2023project}
is a production-ready FLaaS by Microsoft.
Its Android training is marketed as accelerated,
and its iOS training is not mentioned.

% (You can use table to summarize the features of existing methods;
% or you can conduct the comparative study by
% testing some state-of-the-art methods on your selected dataset.)

\begin{table}[htbp]
\caption{Table Type Styles}
\begin{center}
\begin{tabular}{|c|c|c|c|}
\hline
\textbf{Table}&\multicolumn{3}{|c|}{\textbf{Table Column Head}} \\
\cline{2-4} 
\textbf{Head} & \textbf{\textit{Table column subhead}}& \textbf{\textit{Subhead}}& \textbf{\textit{Subhead}} \\
\hline
copy& More table copy$^{\mathrm{a}}$& &  \\
\hline
\multicolumn{4}{l}{$^{\mathrm{a}}$Sample of a Table footnote.}
\end{tabular}
\label{tab1}
\end{center}
\end{table}

\begin{figure}[htbp]
\centerline{
    % \includegraphics{fig1.png}
}
\caption{Example of a figure caption.}
\label{fig}
\end{figure}

\section{Discussion}

% TODO: Provide an outlook on the development of AI technology in
% your research area based on your knowledge of your research area and
% your understanding of AI.
% You can discuss some open challenges and try to
% provide the corresponding potential solutions or
% discuss the potential research directions.

Security remains a critical concern in FL.
Attacks to FL systems to reverse engineer the training data have been
demonstrated to be practical~\cite{sun2019really}.
Mechanism that increase anonymity and
reduce the risk of successful attacks have been proposed and implemented.
For example, differential privacy and secure aggregation are
adopted by many production-level FL frameworks and
their effectiveness has been demonstrated.

However, the real-world application on mobile devices still faces a fundamental
issue of trust.
After the mobile applications obtain the training data,
there is no obvious way to verify whether the data are used for FL,
or are in fact sent to a central server.
Companies may well use FL to cover their direct data collection under the hood.

\printbibliography

% TODO: at least 10 references,
% 50\% references should be published within 5 years,
% blogs/website/news reports should be less than 10\%.

\end{document}
