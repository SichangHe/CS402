\documentclass[conference]{IEEEtran}
\IEEEoverridecommandlockouts
% The preceding line is only needed to identify funding in the first footnote. If that is unneeded, please comment it out.
\usepackage{amsmath,amssymb,amsfonts}
\usepackage{algorithmic}
\usepackage[backend=bibtex,style=ieee,natbib=true]{biblatex}
\usepackage{graphicx}
\usepackage{textcomp}
\usepackage{xcolor}

\addbibresource{main.bib}

\begin{document}

\title{
    COMPSCI 402 --- Artificial Intelligence\\
    Final Report\\
    The Development and Outlook of AI Techniques in the field of
    Machine Learning on Smartphones
}

\author{
    \IEEEauthorblockN{Sichang He}\\
    sichang.he@dukekunshan.edu.cn
}

\maketitle

\begin{abstract}
TODO: Insert a very brief paragraph to summarize your essay
\end{abstract}

\section{Introduction}

TODO: Briefly introduce your understanding of AI,
provide an overview of the application of the AI in your research area.
For example, you can discuss the current development trend and
provide a roadmap of the development of AI techniques in your research area.

\section{TODO: Main Body}

\cite{beutel2020flower}

% (You can use table to summarize the features of existing methods;
% or you can conduct the comparative study by
% testing some state-of-the-art methods on your selected dataset.)

\begin{table}[htbp]
\caption{Table Type Styles}
\begin{center}
\begin{tabular}{|c|c|c|c|}
\hline
\textbf{Table}&\multicolumn{3}{|c|}{\textbf{Table Column Head}} \\
\cline{2-4} 
\textbf{Head} & \textbf{\textit{Table column subhead}}& \textbf{\textit{Subhead}}& \textbf{\textit{Subhead}} \\
\hline
copy& More table copy$^{\mathrm{a}}$& &  \\
\hline
\multicolumn{4}{l}{$^{\mathrm{a}}$Sample of a Table footnote.}
\end{tabular}
\label{tab1}
\end{center}
\end{table}

\begin{figure}[htbp]
\centerline{
    % \includegraphics{fig1.png}
}
\caption{Example of a figure caption.}
\label{fig}
\end{figure}

\section{Discussion}

TODO: Provide an outlook on the development of AI technology in
your research area based on your knowledge of your research area and
your understanding of AI.
You can discuss some open challenges and try to
provide the corresponding potential solutions or
discuss the potential research directions.

\printbibliography

TODO: at least 10 references,
50\% references should be published within 5 years,
blogs/website/news reports should be less than 10\%.

\end{document}
